%-----------------------------------------------------------------------------%
% В отличии от классического TeX использует кодировку UTF-8 для входных файлов,
% что позволяет не заботиться о выборе нужной кодировки и свободно использовать
% спецсимволы и символы иностранных языков доступные в Unicode. XeLaTeX генерирует
% на выходе PDF минуя стадию DVI. Он поддерживает шрифты в форматах TrueType,
% OpenType и AAT, что позволяет использовать в документе большинство современных шрифтов.
%----------------------------------------------------------------------------%

% Ипспользуем тип article, шрифт 14pt
\documentclass[a4paper,14pt,oneside]{extarticle}

% Пакет xltxtra выполняет основные настройки XeLaTeX и загружает пакет fontspec
% необходимый для управления шрифтами. Команда defaultfontfeatures задает
% использование традиционных лигатур. Команды серии set...font задают шрифты документа.
\usepackage{xltxtra}
% включаем лигатуры обычного теха (например '--' создаст нам короткое тире)
\defaultfontfeatures{Ligatures=TeX}

%--------------------------- Локализация под русский язык --------------------%

%\usepackage[T2A]{fontenc}			% Кодировка
%\usepackage[utf8]{inputenc}		% Кодировка исходного текста
\usepackage{polyglossia}            % Локализация
\setmainlanguage{russian}
\setotherlanguage{english}
\setkeys{russian}{babelshorthands=true}
\newfontfamily{\cyrillicfont}{Times New Roman}
\newfontfamily{\cyrillicfontrm}{Times New Roman}
\setromanfont{Times New Roman}
%\setsansfont{Arial}
%\setmonofont{Courier New}

% Пакет fontspec вроде не работает в преамбуле, но пока оставлю
% info: https://tex.stackexchange.com/questions/352804/setmainfont-vs-fontspec
\usepackage{fontspec}
\setmainfont{Times New Roman}       % Times New Roman шрифт

% Переопределим названия рисунков, таблиц и содержания
\addto\captionsrussian{%
  \renewcommand{\figurename}{Рисунок}%
  \renewcommand{\tablename}{Таблица}%
}

\usepackage{geometry}               % Зададим поля
	\geometry{top=20mm}
	\geometry{bottom=20mm}
	\geometry{left=30mm}
	\geometry{right=15mm}

%-------------------- Настройка всяких отступов ------------------------------%

\usepackage{indentfirst}            % Абзацный отступ
\setlength{\parindent}{1.25cm}      % Отступ 1.25 см
\usepackage{setspace}               % Интерлиньяж
\setlength{\parskip}{.5ex}          % Разрыв между абзацами
\onehalfspacing                     % Интерлиньяж 1.5
%\doublespacing                     % Интерлиньяж 2
%\singlespacing                     % Интерлиньяж 1


%---------- Стилизация структурных элементов. разделов и подразделов ---------%
% Для основных структурных элементов используется \part*
% для нумерованных разделов и подразделов используется \section \subsection и тд.
\usepackage[raggedright,explicit]{titlesec}
\titleformat{\part}
    {\bfseries\normalsize\centering\uppercase} % Полужирный, по центру, прописными
    {}                              % Вставка перед названием структ. элемента
    {1em}                           % Отступ после номера раздела
    {#1}                            % Название раздела
    []                              % Вставка после названия раздела
\titlespacing*{\part}{0pt}{-30pt}{\parskip}

\titleformat{\section}[block]
    {\bfseries\normalsize}          % Полужирный
    {\thesection}                   % Вставка перед названием структ. элемента
    {1ex}                           % Отступ после номера раздела
    {#1}                            % Название раздела
    []                              % Вставка после названия раздела
\titlespacing*{\section}{\parindent}{\parskip}{\parskip}

\titleformat{\subsection}[block]
    {\bfseries\normalsize}          % Полужирный
    {\thesubsection}                % Вставка перед названием структ. элемента
    {1ex}                           % Отступ после номера раздела
    {#1}                            % Название раздела
    []                              % Вставка после названия раздела
\titlespacing*{\subsection}{\parindent}{\parskip}{\parskip}

\titleformat{\subsubsection}[block]
    {\normalsize}                   % Полужирный
    {\thesubsubsection}             % Вставка перед названием структ. элемента
    {1ex}                           % Отступ после номера раздела
    {#1}                            % Название раздела
    []                              % Вставка после названия раздела
\titlespacing*{\subsubsection}{\parindent}{\parskip}{\parskip}

%------------------ Стилизация СОДЕРЖАНИЯ ------------------------------------%

% Самое вкусное... На что пришлось потратить большую часть времени, чтобы
% добиться соответствия ГОСТу, поэтому оставил два решения:
% 1) с помощью пакета titletoc, который, как я понял, хардкодит эти значения
%    в стиль документа,
% 2) захардкодил сам без использованя пакета.
% А зачем использовать пакет, если можно сделать без него.

% Содержание titletoc
%\usepackage{titletoc}
%\dottedcontents{part}[1.25cm]{}{2em}{1pc}
%\dottedcontents{section}[1.25cm]{}{2em}{1pc}
%\dottedcontents{subsection}[2.25cm]{}{2em}{1pc}
%\dottedcontents{subsubsection}[1.25cm]{}{2em}{1pc}

\addto\captionsrussian{\renewcommand{\contentsname}{\centering СОДЕРЖАНИЕ}}
% -------------------------------------------------------------^^^^^^^^^^-----%
% -------------------- похоже на хак, но элегантного решения я не нашел (>_<)-%

\makeatletter
\renewcommand{\l@part}{\@dottedtocline{1}{2ex}{2ex}}
\renewcommand{\l@section}{\@dottedtocline{1}{2ex}{2ex}}
\renewcommand{\l@subsection}{\@dottedtocline{1}{4ex}{4ex}}
\renewcommand{\l@subsubsection}{\@dottedtocline{1}{6ex}{6ex}}
\makeatother


%------------------------ Рисунки --------------------------------------------%

\usepackage{graphicx}
% Зададим относительный путь до картинок
\graphicspath{ {./img/} }
% Заменим разделитель на тире и центрируем название
\usepackage[labelsep=endash]{caption}
% Зададим отступы слева и справа для названия картинки
\captionsetup[figure]{justification=centering,margin=2cm}
% Отрегулируем отступы снизу и сверху названия
\captionsetup[figure]{belowskip=-1ex,aboveskip=1.5ex}


%------------------------ Таблицы --------------------------------------------%

%Для длинных таблиц рекомендуют использовать данный пакет
%https://tex.stackexchange.com/questions/59309/ltablex-customize-caption
\usepackage{ltablex}

% Отображаем название таблицы слева
\captionsetup[table]{singlelinecheck=false,justification=raggedright}
% Уменьшим отступ после названия
\captionsetup[table]{aboveskip=.5ex}


%----------------------- Списки ----------------------------------------------%
\usepackage{calc}
\usepackage{enumitem}

\setlist{
    topsep=0pt,                   % отступ сверху и снизу списка
    partopsep=0pt,                % то же самое
    leftmargin=0pt,               % отступ слева
    labelsep=0pt,                 % отступ метки
    align=left,                   % выравнивание метки
    listparindent=\parindent,     % отступ первой строки абзаца
    itemsep=0pt,                  % отступ между элементами
    parsep=0pt                    % отступ между абзацами и элементами
}
\setlist[itemize]{
    label=---~,  %
    labelwidth=1.2em,%
    itemindent=\parindent+\labelwidth%
}
\setlist[enumerate]{
    label=\arabic*),%
    labelwidth=1.4em,%
    itemindent=\parindent+\labelwidth%
}

% Переопределим алфавит asbuk, т.к. он использует буквы з,о и др., которые
% по ГОСТу запрещены
\makeatletter
\def\asbukx#1{\expandafter\@asbukx\csname c@#1\endcsname}
\def\@asbukx#1{\ifcase#1\or a\or б\or в\or г\or д\or е\or ж\or и\or к\or л\or м\or н\or п\or р\or с\or т\or у\or ф\or х\or ц\or ш\or щ\or э\or ю\or я\fi}

\def\Asbukx#1{\expandafter\@Asbukx\csname c@#1\endcsname}
\def\@Asbukx#1{\ifcase#1\or А\or Б\or В\or Г\or Д\or Е\or Ж\or И\or К\or Л\or М\or Н\or П\or Р\or С\or Т\or У\or Ф\or Х\or Ц\or Ш\or Щ\or Э\or Ю\or Я\fi}
\makeatother

% Создадим счетчики
\AddEnumerateCounter{\asbukx}{\@asbukx}{7}
\AddEnumerateCounter{\Asbukx}{\@Asbukx}{7}


% Создадим свой лист
\newlist{asblist}{enumerate}{2}
\setlist[asblist]{
    label=\asbukx*),
    labelwidth=1.4em,
    itemindent=\parindent+\labelwidth
}

\newlist{Asblist}{enumerate}{2}
\setlist[Asblist]{
    label=\Asbukx*),
    labelwidth=1.4em,
    itemindent=\parindent+\labelwidth
}



%---------------------- Часто встречаются данные пакеты ----------------------%
\usepackage{amsmath}
\usepackage{amsfonts}


%---------------------- Библиография -----------------------------------------%
\usepackage[style=russian]{csquotes}
% Подключаем пакет Biblatex
\usepackage[
    backend=biber,%
    sorting=none,%
    style=gost-numeric%
]{biblatex}
\addbibresource{bibliography.bib}

% Уберем точку после источника
\DeclareFieldFormat{labelnumberwidth}{#1}


%----------------------- Макросы ---------------------------------------------%

% Новая команда для создания структурного элемента или раздела без добавления
% в содержание
% https://stackoverflow.com/questions/2785260/hide-an-entry-from-toc-in-latex
\newcommand{\nocontentsline}[3]{}
\newcommand{\tocless}[2]{\bgroup\let\addcontentsline=\nocontentsline#1{#2}\egroup}
% используется так: \tocless\part{имя} или \tocless\(sub)section{имя}

% Удобная вставка картинок
% принимает аргументы: name, scale, caption, label
\newcommand{\fig}[4]{  %

    \begin{figure}[h] %
        \centering    %
        \includegraphics[scale=#2]{#1} %
        \caption{#3}  %
        \label{fig:#4}
    \end{figure}      %
}

% Чтобы постоянно не прописывать разрыв страницы
% создадим свою команду
\newcommand{\structel}[1]{%
    \clearpage%
    \part*{#1}%
}

\newcommand{\sect}[1]{%
    \clearpage%
    \section{#1}%
}

\newcommand{\ssect}[1]{%
    \subsection{#1}%
}

\newcommand{\sssect}[1]{%
    \subsubsection{#1}%
}

% Показать библиографию
\newcommand{\printbib}{%
    % Нативный способ добавления в toc
    %\printbibliography[heading=bibintoc,title={\centering СПИСОК ИСПОЛЬЗОВАННЫХ ИСТОЧНИКОВ}]
    % Но я предпочел сделать так
    \printbibliography[heading=none]
}

% Команда для устранения переносов в содержании, пока не разобрался с ней
%\newcommand{\banhyphens}{\hyphenpenalty=10000\exhyphenpenalty=10000{}}
%\pretocmd{\tableofcontents}{\begingroup\banhyphens}{}{}
