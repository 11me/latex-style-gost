%xelatex
%-----------------------------------------------------------------------------%
% В отличии от классического TeX использует кодировку UTF-8 для входных файлов,
% что позволяет не заботиться о выборе нужной кодировки и свободно использовать
% спецсимволы и символы иностранных языков доступные в Unicode. XeLaTeX генерирует
% на выходе PDF минуя стадию DVI. Он поддерживает шрифты в форматах TrueType,
% OpenType и AAT, что позволяет использовать в документе большинство современных шрифтов.
%----------------------------------------------------------------------------%

% Ипспользуем тип article, шрифт 14pt
\documentclass[a4paper,14pt,oneside]{extarticle}

% Пакет xltxtra выполняет основные настройки XeLaTeX и загружает пакет fontspec
% необходимый для управления шрифтами. Команда defaultfontfeatures задает
% использование традиционных лигатур. Команды серии set...font задают шрифты документа.
\usepackage{xltxtra}
% включаем лигатуры обычного теха (например '--' создаст нам короткое тире)
\defaultfontfeatures{Ligatures=TeX}

%--------------------------- Локализация под русский язык --------------------%

%\usepackage[T2A]{fontenc}			% Кодировка
%\usepackage[utf8]{inputenc}		% Кодировка исходного текста
\usepackage{polyglossia}            % Локализация
\setmainlanguage{russian}
\setotherlanguage{english}
\setkeys{russian}{babelshorthands=true}
\newfontfamily{\cyrillicfont}{Times New Roman}
\newfontfamily{\cyrillicfontrm}{Times New Roman}
\setromanfont{Times New Roman}
%\setsansfont{Arial}
%\setmonofont{Courier New}

% Пакет fontspec вроде не работает в преамбуле, но пока оставлю
% info: https://tex.stackexchange.com/questions/352804/setmainfont-vs-fontspec
\usepackage{fontspec}
\setmainfont{Times New Roman}       % Times New Roman шрифт

% Переопределим названия рисунков, таблиц и содержания
\addto\captionsrussian{%
  \renewcommand{\figurename}{Рисунок}%
  \renewcommand{\tablename}{Таблица}%
}

\usepackage{geometry}               % Зададим поля
	\geometry{top=20mm}
	\geometry{bottom=20mm}
	\geometry{left=30mm}
	\geometry{right=15mm}

%-------------------- Настройка всяких отступов ------------------------------%

\usepackage{indentfirst}            % Абзацный отступ
\setlength{\parindent}{1.25cm}      % Отступ 1.25 см
\usepackage{setspace}               % Интерлиньяж
\setlength{\parskip}{.5ex}          % Разрыв между абзацами
\onehalfspacing                     % Интерлиньяж 1.5
%\doublespacing                     % Интерлиньяж 2
%\singlespacing                     % Интерлиньяж 1


%---------- Стилизация структурных элементов. разделов и подразделов ---------%
% Для основных структурных элементов используется \part*
% для нумерованных разделов и подразделов используется \section \subsection и тд.
\usepackage[raggedright,explicit]{titlesec}
\titleformat{\part}
    {\bfseries\normalsize\centering\uppercase} % Полужирный, по центру, прописными
    {}                              % Вставка перед названием структ. элемента
    {1em}                           % Отступ после номера раздела
    {#1}                            % Название раздела
    []                              % Вставка после названия раздела
\titlespacing*{\part}{0pt}{-30pt}{\parskip}

\titleformat{\section}[block]
    {\bfseries\normalsize}          % Полужирный
    {\thesection}                   % Вставка перед названием структ. элемента
    {1ex}                           % Отступ после номера раздела
    {#1}                            % Название раздела
    []                              % Вставка после названия раздела
\titlespacing*{\section}{\parindent}{\parskip}{\parskip}

\titleformat{\subsection}[block]
    {\bfseries\normalsize}          % Полужирный
    {\thesubsection}                % Вставка перед названием структ. элемента
    {1ex}                           % Отступ после номера раздела
    {#1}                            % Название раздела
    []                              % Вставка после названия раздела
\titlespacing*{\subsection}{\parindent}{\parskip}{\parskip}

\titleformat{\subsubsection}[block]
    {\normalsize}                   % Полужирный
    {\thesubsubsection}             % Вставка перед названием структ. элемента
    {1ex}                           % Отступ после номера раздела
    {#1}                            % Название раздела
    []                              % Вставка после названия раздела
\titlespacing*{\subsubsection}{\parindent}{\parskip}{\parskip}

%------------------ Стилизация СОДЕРЖАНИЯ ------------------------------------%

% Самое вкусное... На что пришлось потратить большую часть времени, чтобы
% добиться соответствия ГОСТу, поэтому оставил два решения:
% 1) с помощью пакета titletoc, который, как я понял, хардкодит эти значения
%    в стиль документа,
% 2) захардкодил сам без использованя пакета.
% А зачем использовать пакет, если можно сделать без него.

% Содержание titletoc
%\usepackage{titletoc}
%\dottedcontents{part}[1.25cm]{}{2em}{1pc}
%\dottedcontents{section}[1.25cm]{}{2em}{1pc}
%\dottedcontents{subsection}[2.25cm]{}{2em}{1pc}
%\dottedcontents{subsubsection}[1.25cm]{}{2em}{1pc}

\addto\captionsrussian{\renewcommand{\contentsname}{\centering СОДЕРЖАНИЕ}}
% -------------------------------------------------------------^^^^^^^^^^-----%
% -------------------- похоже на хак, но элегантного решения я не нашел (>_<)-%

\makeatletter
\renewcommand{\l@part}{\@dottedtocline{1}{2ex}{2ex}}
\renewcommand{\l@section}{\@dottedtocline{1}{2ex}{2ex}}
\renewcommand{\l@subsection}{\@dottedtocline{1}{4ex}{4ex}}
\renewcommand{\l@subsubsection}{\@dottedtocline{1}{6ex}{6ex}}
\makeatother


%------------------------ Рисунки --------------------------------------------%

\usepackage{graphicx}
% Зададим относительный путь до картинок
\graphicspath{ {./img/} }
% Заменим разделитель на тире и центрируем название
\usepackage[labelsep=endash]{caption}
% Зададим отступы слева и справа для названия картинки
\captionsetup[figure]{justification=centering,margin=2cm}
% Отрегулируем отступы снизу и сверху названия
\captionsetup[figure]{belowskip=-1ex,aboveskip=1.5ex}


%------------------------ Таблицы --------------------------------------------%

%Для длинных таблиц рекомендуют использовать данный пакет
%https://tex.stackexchange.com/questions/59309/ltablex-customize-caption
\usepackage{ltablex}

% Отображаем название таблицы слева
\captionsetup[table]{singlelinecheck=false,justification=raggedright}
% Уменьшим отступ после названия
\captionsetup[table]{aboveskip=.5ex}


%----------------------- Списки ----------------------------------------------%
\usepackage{calc}
\usepackage{enumitem}

\setlist{
    topsep=0pt,                   % отступ сверху и снизу списка
    partopsep=0pt,                % то же самое
    leftmargin=0pt,               % отступ слева
    labelsep=0pt,                 % отступ метки
    align=left,                   % выравнивание метки
    listparindent=\parindent,     % отступ первой строки абзаца
    itemsep=0pt,                  % отступ между элементами
    parsep=0pt                    % отступ между абзацами и элементами
}
\setlist[itemize]{
    label=---~,  %
    labelwidth=1.2em,%
    itemindent=\parindent+\labelwidth%
}
\setlist[enumerate]{
    label=\arabic*),%
    labelwidth=1.4em,%
    itemindent=\parindent+\labelwidth%
}

% Переопределим алфавит asbuk, т.к. он использует буквы з,о и др., которые
% по ГОСТу запрещены
\makeatletter
\def\asbukx#1{\expandafter\@asbukx\csname c@#1\endcsname}
\def\@asbukx#1{\ifcase#1\or a\or б\or в\or г\or д\or е\or ж\or и\or к\or л\or м\or н\or п\or р\or с\or т\or у\or ф\or х\or ц\or ш\or щ\or э\or ю\or я\fi}

\def\Asbukx#1{\expandafter\@Asbukx\csname c@#1\endcsname}
\def\@Asbukx#1{\ifcase#1\or А\or Б\or В\or Г\or Д\or Е\or Ж\or И\or К\or Л\or М\or Н\or П\or Р\or С\or Т\or У\or Ф\or Х\or Ц\or Ш\or Щ\or Э\or Ю\or Я\fi}
\makeatother

% Создадим счетчики
\AddEnumerateCounter{\asbukx}{\@asbukx}{7}
\AddEnumerateCounter{\Asbukx}{\@Asbukx}{7}


% Создадим свой лист
\newlist{asblist}{enumerate}{2}
\setlist[asblist]{
    label=\asbukx*),
    labelwidth=1.4em,
    itemindent=\parindent+\labelwidth
}

\newlist{Asblist}{enumerate}{2}
\setlist[Asblist]{
    label=\Asbukx*),
    labelwidth=1.4em,
    itemindent=\parindent+\labelwidth
}



%---------------------- Часто встречаются данные пакеты ----------------------%
\usepackage{amsmath}
\usepackage{amsfonts}


%---------------------- Библиография -----------------------------------------%
\usepackage[style=russian]{csquotes}
% Подключаем пакет Biblatex
\usepackage[
    backend=biber,%
    sorting=none,%
    style=gost-numeric%
]{biblatex}
\addbibresource{bibliography.bib}

% Уберем точку после источника
\DeclareFieldFormat{labelnumberwidth}{#1}


%----------------------- Макросы ---------------------------------------------%

% Новая команда для создания структурного элемента или раздела без добавления
% в содержание
% https://stackoverflow.com/questions/2785260/hide-an-entry-from-toc-in-latex
\newcommand{\nocontentsline}[3]{}
\newcommand{\tocless}[2]{\bgroup\let\addcontentsline=\nocontentsline#1{#2}\egroup}
% используется так: \tocless\part{имя} или \tocless\(sub)section{имя}

% Удобная вставка картинок
% принимает аргументы: name, scale, caption, label
\newcommand{\fig}[4]{  %

    \begin{figure}[h] %
        \centering    %
        \includegraphics[scale=#2]{#1} %
        \caption{#3}  %
        \label{fig:#4}
    \end{figure}      %
}

% Чтобы постоянно не прописывать разрыв страницы
% создадим свою команду
\newcommand{\structel}[1]{%
    \clearpage%
    \part*{#1}%
}

\newcommand{\sect}[1]{%
    \clearpage%
    \section{#1}%
}

\newcommand{\ssect}[1]{%
    \subsection{#1}%
}

\newcommand{\sssect}[1]{%
    \subsubsection{#1}%
}

% Показать библиографию
\newcommand{\printbib}{%
    % Нативный способ добавления в toc
    %\printbibliography[heading=bibintoc,title={\centering СПИСОК ИСПОЛЬЗОВАННЫХ ИСТОЧНИКОВ}]
    % Но я предпочел сделать так
    \printbibliography[heading=none]
}

% Команда для устранения переносов в содержании, пока не разобрался с ней
%\newcommand{\banhyphens}{\hyphenpenalty=10000\exhyphenpenalty=10000{}}
%\pretocmd{\tableofcontents}{\begingroup\banhyphens}{}{}


%-------------------------- ОСНОВНАЯ ЧАСТЬ ДОКУМЕНТА -------------------------%
\begin{document}

\tableofcontents

\structel{ВВЕДЕНИЕ}

5.7.1 Введение должно содержать оценку современного состояния решаемой
научно-технической проблемы, основание и исходные данные для разработки темы,
обоснование необходимости проведения НИР, сведения о планируемом
научно-техническом уровне разработки, о патентных исследованиях и выводы из
них, сведения о метрологическом обеспечении НИР. Во введении должны быть
отражены актуальность и новизна темы, связь данной работы с другими
научно-исследовательскими работами.

5.7.2 Во введении промежуточного отчета по этапу НИР должны быть указаны цели и
задачи исследований, выполненных на данном этапе, их место в выполнении отчета
о НИР в целом.


\sect{Общие требования}
Допускается при подготовке заключительного отчета о НИР печатать через один
интервал, если отчет имеет значительный объем (500 и более страниц). Цвет
шрифта должен быть черным, размер шрифта - не менее 12 пт.  Рекомендуемый тип
шрифта для основного текста отчета - Times New Roman. Полужирный шрифт
применяют только для заголовков разделов и подразделов, заголовков структурных
элементов. Использование курсива допускается для обозначения объектов
(биология, геология, медицина, нанотехнологии, генная инженерия и др.) и
написания терминов (например, in vivo, in vitro) и иных объектов и терминов на
латыни.  Для акцентирования внимания может применяться выделение текста с
помощью шрифта иного начертания, чем шрифт основного текста, но того же кегля и
гарнитуры. Разрешается для написания определенных терминов, формул, теорем
применять шрифты разной гарнитуры. Текст отчета следует печатать, соблюдая
следующие размеры полей: левое - 30 мм, правое - 15 мм, верхнее и нижнее - 20
мм. Абзацный отступ должен быть одинаковым по всему тексту отчета и равен 1,25
см.

\sect{Построение отчета}
6.2.1 Наименования структурных элементов отчета: "СПИСОК ИСПОЛНИТЕЛЕЙ",
"РЕФЕРАТ", "СОДЕРЖАНИЕ", "ТЕРМИНЫ И ОПРЕДЕЛЕНИЯ", "ПЕРЕЧЕНЬ СОКРАЩЕНИЙ И
ОБОЗНАЧЕНИЙ", "ВВЕДЕНИЕ", "ЗАКЛЮЧЕНИЕ",\newline"СПИСОК ИСПОЛЬЗОВАННЫХ ИСТОЧНИКОВ",
"ПРИЛОЖЕНИЕ" служат заголовками структурных элементов отчета.

Заголовки структурных элементов следует располагать в середине строки без точки
в конце, прописными буквами, не подчеркивая. Каждый структурный элемент и
каждый раздел основной части отчета начинают с новой страницы.

6.2.2 Основную часть отчета следует делить на разделы, подразделы и пункты.
Пункты при необходимости могут делиться на подпункты. Разделы и подразделы
отчета должны иметь заголовки. Пункты и подпункты, как правило, заголовков не
имеют.

6.2.3 Заголовки разделов и подразделов основной части отчета следует начинать с
абзацного отступа и размещать после порядкового номера, печатать с прописной
буквы, полужирным шрифтом, не подчеркивать, без точки в конце. Пункты и
подпункты могут иметь только порядковый номер без заголовка, начинающийся с
абзацного отступа.

6.2.4 Если заголовок включает несколько предложений, их разделяют точками.
Переносы слов в заголовках не допускаются.

\sect{Нумерация разделов и подразделов}
6.4.1 Разделы должны иметь порядковые номера в пределах всего отчета,
обозначенные арабскими цифрами без точки и расположенные с абзацного отступа.
Подразделы должны иметь нумерацию в пределах каждого раздела. Номер подраздела
состоит из номеров раздела и подраздела, разделенных точкой. В конце номера
подраздела точка не ставится. Разделы, как и подразделы, могут состоять из
одного или нескольких пунктов.

6.4.2 Если отчет не имеет подразделов, то нумерация пунктов в нем должна быть в
пределах каждого раздела и номер пункта должен состоять из номеров раздела и
пункта, разделенных точкой. В конце номера пункта точка не ставится.

Если отчет имеет подразделы, то нумерация пунктов должна быть в пределах
подраздела и номер пункта должен состоять из номеров раздела, подраздела и
пункта, разделенных точками.
6.4.3 Если раздел или подраздел состоит из одного пункта, то пункт не
нумеруется.

6.4.4 Если текст отчета подразделяется только на пункты, они нумеруются
порядковыми номерами в пределах отчета.

6.4.5 Пункты при необходимости могут быть разбиты на подпункты, которые должны
иметь порядковую нумерацию в пределах каждого пункта: 4.2.1.1, 4.2.1.2, 4.2.1.3
и т.д.

\sect{Перечисления}

Внутри пунктов или подпунктов могут быть приведены перечисления.  Перед каждым
элементом перечисления следует ставить тире. При необходимости ссылки в тексте
отчета на один из элементов перечисления вместо тире ставят строчные буквы
русского алфавита со скобкой, начиная с буквы <<а>> (за исключением букв ё, з,
й, о, ч, ъ, ы, ь). Простые перечисления отделяются запятой, сложные~--- точкой
с запятой.

При наличии конкретного числа перечислений допускается перед каждым элементом
перечисления ставить арабские цифры, после которых ставится скобка.
Перечисления приводятся с абзацного отступа в столбик.

\enquote{Проверяем \enquote{кавычки} внутри кавычек}

\ssect{Пример из госта}
Информационно-сервисная служба для обслуживания удаленных пользователей включает следующие
модули:
\begin{itemize}
\item удаленный заказ,
\item виртуальная справочная служба,
\item виртуальный читальный зал.
\end{itemize}

Работа по оцифровке включала следующие технологические этапы:
\begin{asblist}
    \item первичный осмотр и структурирование исходных материалов,
    \item сканирование документов,
    \item обработка и проверка полученных образов,
    \item структурирование оцифрованного массива,
    \item выходной контроль качества массивов графических образов.
    \item выходной контроль качества массивов графических образов.
    \item выходной контроль качества массивов графических образов.
    \item выходной контроль качества массивов графических образов.
    \item выходной контроль качества массивов графических образов.
    \item выходной контроль качества массивов графических образов.
    \item выходной контроль качества массивов графических образов.
\end{asblist}

Камеральные и лабораторные исследования включали разделение всего выявленного
видового состава растений на четыре группы по степени использования их копытными:
\begin{enumerate}
    \item случайный корм,
    \item второстепенный корм,
    \item дополнительный корм,
    \item основной корм.
\end{enumerate}

Разрабатываемое сверхмощное устройство можно будет применять в различных
отраслях реального сектора экономики:
\begin{itemize}
    \item в машиностроении:
    \begin{enumerate}
        \item для очистки отливок от формовочной смеси;
        \item для очистки лопат ок т урбин авиационных двигателей;
        \item для холодной штамповки из лист а;
    \end{enumerate}
    \item в ремонте техники:
    \begin{enumerate}
        \item устранение наслоений на внут ренних стенках труб;
        \item очистка каналов и отверстий небольшого диаметра от грязи.
    \end{enumerate}
\end{itemize}

\sect{Работа с рисунком}

6.5.1 Иллюстрации (чертежи, графики, схемы, компьютерные распечатки, диаграммы,
фотоснимки) следует располагать в отчете непосредственно после текста отчета,
где они упоминаются впервые, или на следующей странице (по возможности ближе к
соответствующим частям текста отчета). На все иллюстрации в отчете должны быть
даны ссылки. При ссылке необходимо писать слово \enquote{рисунок} и его номер,
например: \enquote{в соответствии с рисунком 2} и т.д.

\emph{Пример}. На рисунке~\ref{fig:cat} изображен задумчивый котик, который
с огромным интересом наблюдает за\ldots

\fig{cat_laptop}{.8}{Задумчивый котик}{cat}

6.5.4 Иллюстрации, за исключением иллюстраций, приведенных в приложениях,
следует нумеровать арабскими цифрами сквозной нумерацией. Если рисунок один, то
он обозначается: Рисунок 1.

6.5.5 Иллюстрации каждого приложения обозначают отдельной нумерацией арабскими
цифрами с добавлением перед цифрой обозначения приложения: Рисунок А.3.

6.5.6 Допускается нумеровать иллюстрации в пределах раздела отчета. В этом
случае номер иллюстрации состоит из номера раздела и порядкового номера
иллюстрации, разделенных точкой: Рисунок 2.1.

6.5.7 Иллюстрации при необходимости могут иметь наименование и пояснительные
данные (подрисуночный текст). Слово "Рисунок", его номер и через тире
наименование помещают после пояснительных данных и располагают в центре под
рисунком без точки в конце.

\sect{Работа с таблицами}
6.6.1 Цифровой материал должен оформляться в виде таблиц. Таблицы применяют для
наглядности и удобства сравнения показателей.

6.6.2 Таблицу следует располагать непосредственно после текста, в котором она
упоминается впервые, или на следующей странице.

На все таблицы в отчете должны быть ссылки. При ссылке следует печатать слово
\enquote{таблица} с указанием ее номера.

6.6.3 Наименование таблицы, при его наличии, должно отражать ее содержание,
быть точным, кратким. Наименование следует помещать над таблицей слева, без
абзацного отступа в следующем формате: Таблица Номер таблицы - Наименование
таблицы. Наименование таблицы приводят с прописной буквы без точки в конце.

Если наименование таблицы занимает две строки и более, то его следует
записывать через один межстрочный интервал.

Таблицу с большим количеством строк допускается переносить на другую страницу.
При переносе части таблицы на другую страницу слово "Таблица", ее номер и
наименование указывают один раз слева над первой частью таблицы, а над другими
частями также слева пишут слова \enquote{Продолжение таблицы} и указывают номер
таблицы.

При делении таблицы на части допускается ее головку или боковик заменять
соответственно номерами граф и строк. При этом нумеруют арабскими цифрами графы
и (или) строки первой части таблицы.

6.6.4 Таблицы, за исключением таблиц приложений, следует нумеровать арабскими
цифрами сквозной нумерацией.

Таблицы каждого приложения обозначаются отдельной нумерацией арабскими цифрами
с добавлением перед цифрой обозначения приложения. Если в отчете одна таблица,
она должна быть обозначена \enquote{Таблица 1} или \enquote{Таблица А.1} (если она приведена в
приложении А).

Допускается нумеровать таблицы в пределах раздела при большом объеме отчета. В
этом случае номер таблицы состоит из номера раздела и порядкового номера
таблицы, разделенных точкой: Таблица 2.3.

6.6.5 Заголовки граф и строк таблицы следует печатать с прописной буквы, а
подзаголовки граф - со строчной буквы, если они составляют одно предложение с
заголовком, или с прописной буквы, если они имеют самостоятельное значение. В
конце заголовков и подзаголовков таблиц точки не ставятся. Названия заголовков
и подзаголовков таблиц указывают в единственном числе.

6.6.6 Таблицы слева, справа, сверху и снизу ограничивают линиями. Разделять
заголовки и подзаголовки боковика и граф диагональными линиями не допускается.
Заголовки граф выравнивают по центру, а заголовки строк - по левому краю.

\begin{table}[h]
    \caption{Какие-то данные}\label{tab:t1}
    \begin{tabular}{|c|c|c|}

        \hline
        h1 & h2 & h3 \\
        \hline
        data & data & data \\
        \hline

    \end{tabular}
\end{table}

\begin{table}[h]
    \caption{Таблица с длинным названием с очень длинным названием и его нужно %
              печать через один интервал}\label{tab:t2}
    \begin{tabular}{|c|c|m{5cm}|}\hline

        h1 & h2 & эта колонка имеет длина 5 см \\\hline
        data & data & data \\\hline

    \end{tabular}
\end{table}

Горизонтальные и вертикальные линии, разграничивающие строки таблицы,
допускается не проводить, если их отсутствие не затрудняет пользование
таблицей.

Добавим еще маленький абзац, чтобы таблица уехала вниз с продолжением.


\begin{tabularx}{\textwidth} {%
% Нужно разметить столбцы, новая фича X, которая как |c|r| и тд в LaTeX
    |>{\small\hsize=0.35\hsize\raggedright\arraybackslash}X
    |>{\small\hsize=0.65\hsize\raggedright\arraybackslash}X|}

% Как же таблица может быть без названия, нужно её непременно обозвать
\caption{Название длинной таблицы}\label{tab:t3} \\

% Обзываем шапку таблицы
\hline
\textbf{Первый столбец} & \textbf{Второй столбец} \\
\endfirsthead

% Эта шапка без \hline будет выглядеть как просто текст
\caption*{Продолжение таблицы~\ref{tab:t3}} \\
\endhead

% Осталось только заполнить таблицу
\hline
Первый столбец первой строки & Тут должны быть результаты проделанной работы. Данные данные
              данные. Тут должны быть результаты проделанной работы. Данные данные
              данные \\
\hline
Первый столбец второй строки & Тут должны быть результаты проделанной работы. Данные данные
              данные \\
\hline
Первый столбец третьей строки & Продолжаем длинную таблицу уже на новой странице!
              нужно больше текста, чтобы таблица оказалась внизу. А этого текста
              еще недостаточно. Вот теперь думаю достаточно. \\
\hline
\multicolumn{2}{|>{\small\raggedright\arraybackslash}X|}%
{Можно даже добавить целую строку!} \\
\hline
А это пятый столбец! & С новыми данными! \\
\hline
\end{tabularx}

6.6.7 Текст, повторяющийся в строках одной и той же графы и состоящий из
одиночных слов, заменяют кавычками. Ставить кавычки вместо повторяющихся цифр,
буквенно-цифровых обозначений, знаков и символов не допускается.  Если текст
повторяется, то при первом повторении его заменяют словами \enquote{то же}, а
далее кавычками.  В таблице допускается применять размер шрифта меньше, чем в
тексте отчета.


\sect{Работа с библиографией и ссылки}
6.9.1 В отчете о НИР рекомендуется приводить ссылки на использованные
источники. При нумерации ссылок на документы, использованные при составлении
отчета, приводится сплошная нумерация для всего текста отчета в целом или для
отдельных разделов. Порядковый номер ссылки (отсылки) приводят арабскими
цифрами в квадратных скобках в конце текста ссылки. Порядковый номер
библиографического описания источника в списке использованных источников
соответствует номеру ссылки.

6.9.2 Ссылаться следует на документ в целом или на его разделы и приложения.

6.9.3 При ссылках на стандарты и технические условия указывают их обозначение,
при этом допускается не указывать год их утверждения при условии полного
описания стандарта и технических условий в списке использованных источников в
соответствии с ГОСТ 7.1.
Попробуем процитировать~\cite{zakas}. И сразу двух
~\cite{deridder}-\cite{antopol}.



\structel{ЗАКЛЮЧЕНИЕ}
Заключение должно содержать:
\begin{itemize}
    \item краткие выводы по результатам выполненной НИР или отдельных ее этапов;
    \item оценку полноты решений поставленных задач;
    \item разработку рекомендаций и исходных данных по конкретному использованию
          результатов НИР;
    \item результаты оценки технико-экономической эффективности внедрения;
    \item результаты оценки научно-технического уровня выполненной НИР в сравнении с
          лучшими достижениями в этой области.
\end{itemize}


\structel{СПИСОК ИСПОЛЬЗОВАННЫХ ИСТОЧНИКОВ}
\printbib

\end{document}
%----------------------------- КОНЕЦ ДОКУМЕНТА -------------------------------%
